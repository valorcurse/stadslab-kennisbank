\chapter{Introductie}

Dit verslag is geschreven met betrekking tot het vak ICT-lab (TIRLAB01 en TIRLAB02). Het is een project aangeboden door het Stadslab Rotterdam. Het Stadslab is een initiatief van de Hogeschool Rotterdam en de Gemeente Rotterdam om iedereen een kans te bieden om met technische apparatuur te werken waar ze normaal geen toegang tot zouden hebben. De enige voorwaarde dat het Stadslab stelt voor het benutten van hun apparatuur is dat de kennis dat opgedaan is tijdens het verblijf vrij wordt gegeven. Hierdoor kunnen andere personen, dat later bij de Stadslab aan hun eigen projecten gaan werken, van dezelfde kennis benutten. Deze kennis kan in vele vormen voorkomen, maar de meest belangrijke zijn de gebruikte instellingen van de gebruikte apparatuur. Dit omdat het uittesten van verschillende instellingen heel veel tijd kan kosten zonder een uitgangspunt.


De doelen van dit project zijn als volgt:
\begin {itemize}
\item Het delen van kennis zo simpel, snel en makkelijk mogelijk te maken.
\item Het doorzoekbaar maken van de gedeelde informatie.	
\item Een overzicht cre\"eren van het aantal bezoeken en welke apparaten zijn gebruikt in een bepaalde aanpasbare periode.
\item Het project goed documenteren zodat andere ontwikkelaars er met zo min mogelijk moeite verder aan kunnen werken.
\end {itemize}