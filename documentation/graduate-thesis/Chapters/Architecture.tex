\chapter{Architectuur}

Het systeem is ontworpen en ge\"implementeerd in de vorm van een webapplicatie omdat het overal makkelijk te bereiken is en werkt op elke systeem met een webbrowser. De applicatie is verdeeld in meerdere onderdelen en elke onderdeel zal besproken worden in deze hoofdstuk.

\section{Checkins en checkouts}

\'E\'en van de onderdelen van de applicatie is de checkins en checkouts gedeelte. Dit gedeelte wordt gebruik om gebruikers de mogelijkheid te geven om in te checken als ze binnen komen en uit te checken als ze het lab verlaten.

\subsection{Checkin}

Als iemand het lab wilt gebruiken, worden ze verzocht om bij de ingang van het Stadslab in te checken. Het inchecken bestaat uit persoonlijk informatie en de doel van het bezoek invullen. De informatie die ingevuld moet worden verschilt per de volgende categori\"en: Student, Bedrijf en Overige.

\subsection{Checkout}

Nadat iemand zijn project heeft afgerond, wordt van diegene dan verwacht om uit te checken voor het verlaten van het lab. Bij het uitchecken moet de gebruiker informatie delen over waarmee en hoe het project is gemaakt.

\section{FabTool}

Het andere onderdeel van de webapplicatie is de FabTool. Hier kunnen gebruikers naar project zoeken en bekijken en kunnen beheerders een overzichten krijgen over hoeveel mensen het lab hebben bezoekt en welke apparaten ze hebben gebruikt.

\subsection{Projecten}

Hier is een lijst van alle projecten (ook bekend als checkouts). Je kan de project door meerdere categori\"en filteren en je kan ze bekijken.

\subsection{Administratie}

\subsection{Aanpassingen}